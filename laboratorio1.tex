\documentclass[12pt,letterpaper]{article}
\usepackage[utf8]{inputenc}
\usepackage[spanish]{babel}
\usepackage{graphicx}
\usepackage[left=2cm,right=2cm,top=2cm,bottom=2cm]{geometry}
\usepackage{graphicx} % figuras
% \usepackage{subfigure} % subfiguras
\usepackage{float} % para usar [H]
\usepackage{amsmath}
%\usepackage{txfonts}
\usepackage{stackrel} 
\usepackage{multirow}
\usepackage{enumerate} % enumerados
\renewcommand{\labelitemi}{$-$}
\renewcommand{\labelitemii}{$\cdot$}
% \author{}
% \title{Caratula}
\begin{document}

% Fancy Header and Footer
% \usepackage{fancyhdr}
% \pagestyle{fancy}
% \cfoot{}
% \rfoot{\thepage}
%

% \usepackage[hidelinks]{hyperref} % CREA HYPERVINCULOS EN INDICE

% \author{}
\title{Caratula}

\begin{titlepage}
\begin{center}
\large{UNIVERSIDAD PRIVADA-DE-TACNA }
\\FACULTAD DE INGENIERIA
\\ESCUELA PROFESIONAL DE INGENIERIA DE SISTEMAS

\vspace*{-0.025in}
\begin{figure}[htb]
\begin{center}

\end{center}
\end{figure}
\begin{center}
    \includegraphics[width=5cm, height=5cm]{img/upt.png}  
\end{center}

\vspace*{0.15in}
INGENIERIA DE SISTEMAS  \\

\vspace*{0.5in}
\begin{large}
TITULO:\\
\end{large}

\vspace*{0.1in}
\begin{Large}
\textbf{LABORATORIO N° 01} \\
\end{Large}

\vspace*{0.3in}
\begin{Large}
\textbf{CURSO:} \\
\end{Large}

\vspace*{0.1in}
\begin{large}
CALIDAD Y PRUEBAS DE SOFTWARE\\
\end{large}

\vspace*{0.3in}
\begin{Large}
\textbf{DOCENTE:} \\
\end{Large}

\vspace*{0.1in}
\begin{large}
Ing. PATRICK CUADROS QUIROGA\\
\end{large}

\vspace*{0.2in}
\vspace*{0.1in}
\begin{large}
Alumno: \\
\begin{flushleft}
Velasquez Garcia, Bryam Michaell		\hfill	(2015053899) \\

\end{flushleft}
\end{large}
\vspace*{0.1in}
\begin{large}
TACNA-PERU \\
\end{large}
\vspace*{0.1in}
\begin{large}
2020\\
\end{large}

\end{center}

\end{titlepage}

\tableofcontents % INDICE
\thispagestyle{empty} % INDICE SIN NUMERO
\newpage
\setcounter{page}{1} % REINICIAR CONTADOR DE PAGINAS DESPUES DEL INDICE


\section {Desarrollo de la practica} 
\section{Descargar SonarQube} 
\textit{docker pull sonarqube}
\begin{center}
    \includegraphics[width=18cm, height=5cm]{img/1.png}  
\end{center}

\section{Ejecutar la  instancia de SonarQube} 
\textit{docker run -d --name sonarqube -p 9000:9000 sonarqube}

        \begin{center}
            \includegraphics[width=18cm, height=3cm]{img/imagen2.png}  
        \end{center}


\newpage
\section{Ingresar al portal con las credenciales} 
\textit{http://localhost:9000/
\\-user: admin
\\-pass:admin}
        \begin{center}
            \includegraphics[width=18cm, height=10cm]{img/imagen3.png}  
        \end{center}

       

\section{Crear una nueva aplicación con el nombre aplicacionNetCore}
        \begin{center}
            \includegraphics[width=18cm, height=3cm]{img/imagen4.png}  
        \end{center}
\section{Generar el token de la nueva aplicación aplicacionNetCore, debera devolver algo similar a:}
    \textit{8a15d2a89c8636f15eb32ebee0993b8d16bff94e}
        \begin{center}
            \includegraphics[width=18cm, height=9cm]{img/imagen5.png}  
        \end{center}
\newpage
\section{Decargar Net Core e instalar}
\textit{https://dotnet.microsoft.com/download/dotnet-core/thank-you/sdk-3.1.300-windows-x64-installer}
        \begin{center}
            \includegraphics[width=18cm, height=9cm]{img/imagen6.png}  
        \end{center}

\section{En un terminal ejecutar e instalar sonar-scanner} 

\textit{dotnet tool install --global dotnet-sonarscanner}
        \begin{center}
            \includegraphics[width=18cm, height=7cm]{img/imagen7.png}  
        \end{center}

\section{En un terminal, acceder a una ruta donde creara una nueva aplicación} 
\textit{dotnet new sln -o aplicacionNetCore
\\cd aplicacionNetCore
\\dotnet new console
\\dotnet sln aplicacionNetCore.sln add aplicacionNetCore.csproj}
        \begin{center}
            \includegraphics[width=18cm, height=9cm]{img/imagen8.png}  
        \end{center}


\section{En el mismo terminal, iniciar la sesión de revisión de sonarqube} 
\textit{dotnet sonarscanner begin /d:sonar.host.url="http://localhost:9000" /d:sonar.login=admin /d:sonar.password=admin /k:”aplicacionNetCore”}
        \begin{center}
                    \includegraphics[width=18cm, height=5cm]{img/imagen10.png}  
        \end{center}

\section{Empezar a  Compilar la aplicación} 
\textit{dotnet build}
        \begin{center}
            \includegraphics[width=18cm, height=6cm]{img/imagen11.png}  
        \end{center}

\newpage
\section{Cerramos la sesión} 
\textit{dotnet sonarscanner end /d:sonar.login=admin /d:sonar.password=admin}
        \begin{center}
                    \includegraphics[width=18cm, height=5cm]{img/cerrarsesion.png}  
        \end{center}

\end{document}
\textbf{}